% Options for packages loaded elsewhere
\PassOptionsToPackage{unicode}{hyperref}
\PassOptionsToPackage{hyphens}{url}
%
\documentclass[
]{article}
\usepackage{amsmath,amssymb}
\usepackage{iftex}
\ifPDFTeX
  \usepackage[T1]{fontenc}
  \usepackage[utf8]{inputenc}
  \usepackage{textcomp} % provide euro and other symbols
\else % if luatex or xetex
  \usepackage{unicode-math} % this also loads fontspec
  \defaultfontfeatures{Scale=MatchLowercase}
  \defaultfontfeatures[\rmfamily]{Ligatures=TeX,Scale=1}
\fi
\usepackage{lmodern}
\ifPDFTeX\else
  % xetex/luatex font selection
\fi
% Use upquote if available, for straight quotes in verbatim environments
\IfFileExists{upquote.sty}{\usepackage{upquote}}{}
\IfFileExists{microtype.sty}{% use microtype if available
  \usepackage[]{microtype}
  \UseMicrotypeSet[protrusion]{basicmath} % disable protrusion for tt fonts
}{}
\makeatletter
\@ifundefined{KOMAClassName}{% if non-KOMA class
  \IfFileExists{parskip.sty}{%
    \usepackage{parskip}
  }{% else
    \setlength{\parindent}{0pt}
    \setlength{\parskip}{6pt plus 2pt minus 1pt}}
}{% if KOMA class
  \KOMAoptions{parskip=half}}
\makeatother
\usepackage{xcolor}
\usepackage[margin=1in]{geometry}
\usepackage{graphicx}
\makeatletter
\def\maxwidth{\ifdim\Gin@nat@width>\linewidth\linewidth\else\Gin@nat@width\fi}
\def\maxheight{\ifdim\Gin@nat@height>\textheight\textheight\else\Gin@nat@height\fi}
\makeatother
% Scale images if necessary, so that they will not overflow the page
% margins by default, and it is still possible to overwrite the defaults
% using explicit options in \includegraphics[width, height, ...]{}
\setkeys{Gin}{width=\maxwidth,height=\maxheight,keepaspectratio}
% Set default figure placement to htbp
\makeatletter
\def\fps@figure{htbp}
\makeatother
\setlength{\emergencystretch}{3em} % prevent overfull lines
\providecommand{\tightlist}{%
  \setlength{\itemsep}{0pt}\setlength{\parskip}{0pt}}
\setcounter{secnumdepth}{-\maxdimen} % remove section numbering
\ifLuaTeX
  \usepackage{selnolig}  % disable illegal ligatures
\fi
\IfFileExists{bookmark.sty}{\usepackage{bookmark}}{\usepackage{hyperref}}
\IfFileExists{xurl.sty}{\usepackage{xurl}}{} % add URL line breaks if available
\urlstyle{same}
\hypersetup{
  pdftitle={Accelerated survival analysis},
  hidelinks,
  pdfcreator={LaTeX via pandoc}}

\title{Accelerated survival analysis}
\author{}
\date{\vspace{-2.5em}}

\begin{document}
\maketitle

Os modelos de regressão de sobrevivência de classe acelerada são um
grupo de técnicas estatísticas usadas para analisar dados em que a
variável de resultado é o tempo até um evento, também conhecidos como
dados de sobrevivência. Estes modelos diferem dos modelos de
sobrevivência padrão por incorporarem o efeito das covariáveis na
própria escala de tempo, em vez de apenas na taxa de risco. Aqui está
uma análise dos três tipos principais (AFT, AH, AO):

\hypertarget{modelo-de-tempo-de-falha-acelerado-aft-model}{%
\subsection{1 - Modelo de tempo de falha acelerado (AFT
model)}\label{modelo-de-tempo-de-falha-acelerado-aft-model}}

O \emph{AFT model} é uma abordagem paramétrica (ou semiparamétrica) que
oferece uma opção robusta de regressão em análise de sobrevivência. A
ideia central por trás dos modelos AFT é que as covariáveis atuam
multiplicativamente no log do tempo até a ocorrência de algum evento de
interesse, acelerando ou desacelerando o processo de falha (KALBFLEISCH;
PRENTICE, 2002). Assim, para a \(i\)-ésima observação, um \emph{AFT
model} é caracterizado por sugerir que \[
\log(T_i) = \mathbf{x}_i^{\top}\pmb{\beta} + \nu_i, \ \ \ \ i=1, \dots, n,
\] em que

\begin{itemize}
\tightlist
\item
  \(T_i\): \(i\)-ésima observação da variável aleatória que define o
  tempo de falha (\(T>0\));
\item
  \(\mathbf{x}=(x_1, \dots, x_k)^{\top}\): Vetor de covariáveis com
  dimensão \(k\) para a \(i\)-ésima observação;
\item
  \(\pmb{\beta}=(\beta_1, \dots, \beta_k)^{\top}\): Vetor com \(k\)
  coeficientes de regressão, onde \(\beta_j\), \(j=1, \dots, k\), indica
  como a covariável \(x_j\) afeta o tempo de sobrevivência logarítmico.
  Um valor \(\beta_j\) positivo sugere que a covariável acelera a falha
  (diminui o tempo de sobrevivência), enquanto um valor negativo implica
  desaceleração;
\item
  \(\nu_i\): Termo de erro aleatório associado a cada observação.
\end{itemize}

Com isso, podemos notar que

\[
T_i = e^{\mathbf{x}_i^{\top}\pmb{\beta}}T_{0i} \ \ \rightarrow \ \ \nu_i = \log(T_{0i}),
\] para o tempo de vida não moderado \(T_0\) distribuído
independentemente de covariáveis (\emph{baseline}), exceto pela sua
própria igualdade. Diferentes distribuições de \(\nu\) implicam
distribuições diferentes de \(T_0\) (GEORGE; SEALS; ABAN, 2014).

Com essas definições, pode-se denotar o \emph{AFT model} em termos de
funções:

\textbf{Survival function}

A função de sobrevivência de \(T|\mathbf{x}\) pode ser descrita pela
função de sobrevivência de \(T_0\): \[
\begin{align*} 
\mathcal{S}(t| \ \pmb{\vartheta}, \pmb{\beta}, \mathbf{x}) & = P_{\pmb{\vartheta}, \pmb{\beta}}(T > t \ |\mathbf{x}) \\
& = P_{\pmb{\vartheta}}\left(e^{\mathbf{x}^{\top}\pmb{\beta}}e^{\nu} > t \right) \\
& = P_{\pmb{\vartheta}}\left(T_0 > te^{-\mathbf{x}^{\top}\pmb{\beta}} \right) \\
& = \mathcal{S}_0\left(te^{-\mathbf{x}^{\top}\pmb{\beta}}\Big|\pmb{\vartheta} \right),
\end{align*}
\] sendo \(\pmb{\vartheta}\) um vetor de parâmetros associado à
distribuição \emph{baseline}. Dessa forma, é possível expressar as
outras funções para o \emph{AFT model}.

\textbf{Density function} \[
\begin{align*} 
f(t| \ \pmb{\vartheta}, \pmb{\beta}, \mathbf{x}) & = -\frac{\partial}{\partial t}\mathcal{S}(t| \ \pmb{\vartheta}, \pmb{\beta}, \mathbf{x}) \\
& = -\frac{\partial}{\partial t}\mathcal{S}_0\left(te^{-\mathbf{x}^{\top}\pmb{\beta}}\Big|\pmb{\vartheta} \right) \\
& = -\frac{\partial}{\partial t}\left[1-F_0\left(te^{-\mathbf{x}^{\top}\pmb{\beta}}\Big|\pmb{\vartheta} \right) \right] \\
& = \frac{\partial}{\partial t}F_0\left(te^{-\mathbf{x}^{\top}\pmb{\beta}}\Big|\pmb{\vartheta} \right) \\
& = f_0\left(te^{-\mathbf{x}^{\top}\pmb{\beta}}\Big|\pmb{\vartheta} \right)e^{-\mathbf{x}^{\top}\pmb{\beta}},
\end{align*}
\] com \(F_0(\cdot|\pmb{\vartheta})\) a função de distribuição
\emph{baseline}.

\textbf{Hazard function} \[
\begin{align*} 
h(t| \ \pmb{\vartheta}, \pmb{\beta}, \mathbf{x}) & = \frac{f(t| \ \pmb{\vartheta}, \pmb{\beta}, \mathbf{x})}{\mathcal{S}(t| \ \pmb{\vartheta}, \pmb{\beta}, \mathbf{x})} \\
& = \frac{f_0\left(te^{-\mathbf{x}^{\top}\pmb{\beta}}\Big|\pmb{\vartheta} \right)e^{-\mathbf{x}^{\top}\pmb{\beta}}}{\mathcal{S}_0\left(te^{-\mathbf{x}^{\top}\pmb{\beta}}\Big|\pmb{\vartheta} \right)} \\
& = h_0\left(te^{-\mathbf{x}^{\top}\pmb{\beta}}\Big|\pmb{\vartheta} \right)e^{-\mathbf{x}^{\top}\pmb{\beta}}
\end{align*}
\]

\textbf{Cumulative hazard function} \[
\begin{align*} 
H(t| \ \pmb{\vartheta}, \pmb{\beta}, \mathbf{x}) & = -\log\mathcal{S}(t| \ \pmb{\vartheta}, \pmb{\beta}, \mathbf{x}) \\
& = -\log\mathcal{S}_0\left(te^{-\mathbf{x}^{\top}\pmb{\beta}}\Big|\pmb{\vartheta} \right) \\
& = H_0\left(te^{-\mathbf{x}^{\top}\pmb{\beta}}\Big|\pmb{\vartheta} \right).
\end{align*}
\]

Os modelos AFT são uma ferramenta valiosa para analisar dados quando há
covariáveis omitidas ou quando a distribuição de probabilidade
subjacente é incerta (BURZYKOWSKI, 2022).

\textbf{References:}

\begin{itemize}
\tightlist
\item
  KALBFLEISCH, John D.; PRENTICE, Ross L. \textbf{The statistical
  analysis of failure time data}. John Wiley \& Sons, 2002.
\item
  BURZYKOWSKI, Tomasz. Semi‐parametric accelerated failure‐time model: A
  useful alternative to the proportional‐hazards model in cancer
  clinical trials. \textbf{Pharmaceutical Statistics}, v. 21, n.~2,
  p.~292-308, 2022.
\item
  GEORGE, Brandon; SEALS, Samantha; ABAN, Inmaculada. Survival analysis
  and regression models. \textbf{Journal of nuclear cardiology}, v. 21,
  n.~4, p.~686-694, 2014.
\end{itemize}

\end{document}
